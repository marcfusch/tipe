%Document presentation
\documentclass[12pt, french]{article}
\usepackage[margin=0.7in]{geometry}

%Custom language
\usepackage[french]{babel} 
\selectlanguage{french}
\usepackage[T1]{fontenc}
\usepackage[utf8]{inputenc}

%Maths packages
\usepackage{esvect}
\usepackage{amsmath} 
\usepackage{amsfonts}
\usepackage{gensymb}
\usepackage{tikz,tkz-tab}
\usepackage{listings}

%\usepackage{eecs}

\definecolor{codegreen}{rgb}{0,0.6,0}
\definecolor{codegray}{rgb}{0.5,0.5,0.5}

\definecolor{codepurple}{rgb}{0.58,0,0.82}
\definecolor{backcolour}{rgb}{0.95,0.95,0.92}

\lstdefinestyle{mystyle}{
    backgroundcolor=\color{backcolour},   
    commentstyle=\color{codegreen},
    keywordstyle=\color{magenta},
    numberstyle=\tiny\color{codegray},
    stringstyle=\color{codepurple},
    basicstyle=\ttfamily\footnotesize,
    breakatwhitespace=false,         
    breaklines=true,                 
    captionpos=b,                    
    keepspaces=true,                 
    numbers=left,                    
    numbersep=5pt,                  
    showspaces=false,                
    showstringspaces=false,
    showtabs=false,                  
    tabsize=5
}

\lstset{style=mystyle}
%CUSTOM COMMANDS
\newcommand{\motclefs}[2]{
\section*{Mots-clefs}
\begin{description}
\item[Mots-clefs] -- #1 
\item[Keywords]   -- #2
\end{description}
}

\newcommand{\positionnementThematique}[1]{
\section*{Positionnement thématique}
{\it #1}}

\newcommand{\Titre}[1]{
\noindent
\textcolor{black}{\Huge\textbf{\textsf{#1}}}
\vspace*{0.75cm}
}
%Custom paragraphs
\usepackage{ulem}
\usepackage{titlesec}
\titleformat{\paragraph}[hang]{\bfseries}{}{0pt}{\uline}
\titlespacing*{\paragrahepage}

%DOCUMENT STARTS%
\begin{document}

\Titre{Réparation de jouet à la lumière de l'impression 3D.}

\positionnementThematique{Sciences de l'ingénieur (Génie Mécanique)}

\motclefs{Impression 3D -- Resistance des matériaux -- Motifs -- Fabrication -- Essai destructif}{3D Printing -- Material strength -- Patterns -- Manufacturing -- Destructive testing}

\section*{Bibliographie commentée}

Les athlètes ont besoin de matériel de plus en plus performant, résistant et moins contraignant.
Depuis le développement de la fabrication par impression 3D, on observe de plus en plus de pièces destinées à l'usage compétitif de haut niveau. \cite{sports} \\

La fabrication par dépôt de matière est très répandue dans ce domaine car la rapidité de réalisation permet un prototypage rapide et peu coûteux \cite{sports}.\\

 Ainsi, cette méthode de fabrication a aussi pu gagner en popularité auprès du grand public. Elle permet la réparation rapide et peu coûteuse \cite{moula} d'objets dont la qualité de fabrication a diminué alors que la difficulté de réparation a augmenté. De fait, les jeux et jouets sont eux passés dans cette catégorie de "consommables".  \\
 
 L'impression 3D semble donc être une bonne solution de réparation.\cite{toys}. Toutefois, des paramètres comme le motif et la densité de remplissage doivent être spécifiés avant d'envoyer les instructions à la machine.
  Il paraît donc logique que ces derniers aient une influence sur le résultat final.\\ Dans ces conditions, comment quantifier ces paramètres?\\
 
Pour tester la résistance d'une pièce, on procède à un essai de traction \cite{tensiletestd638}. Le protocole expérimental ainsi que la pièce étudiée doivent donc suivre des normes internationales. Nous utiliserons la norme de traction de plastiques ASTM-D638 \cite{ASTMD638}.\\

 Les essais nous donneront des graphes de déformations. Ces derniers représentent la contrainte appliquée à l'éprouvette en fonction de son allongement. On peut donc en déduire la pente qui est le module de Young (E) \cite{rdm} ainsi qu'un point caractéristique qui est la contrainte à la rupture.\\ 

Ainsi, en générant différents motifs (à l'aide d'un logiciel spécifique \cite{prusa}) dans des éprouvettes de test, nous procéderons à un premier essai de traction avec la machine \cite{instron}. Nous pourrons donc établir un premier lien entre le motif et le rapport résistance masse.\\
Ensuite, nous ferons varier la densité du motif retenu afin d'obtenir un second lien entre la densité du motif et le rapport résistance masse. \cite{infill}\\

	 Il nous sera ainsi possible d'extrapoler une configuration optimale alliant solidité et légèreté. Nous imprimerons une pièce pour réparer un jouet. Opération qui, tout en évitant une mise au rebut, améliorera la conception du jouet.
\section*{Problématique retenue}
 Quels paramètres de remplissage utiliser lors de l'impression 3D d'une pièce en PLA afin d'optimiser sa résistance sans sacrifier la contrainte de masse? Application à une réparation de jouet.
\section*{Objectifs du TIPE}
\begin{enumerate}
	\item	Modélisation: Réaliser le plan d'une éprouvette en 3D respectant des standards internationaux.
    \item	Réalisation: Impression et optimisation du processus d’impression afin d’obtenir le plus d’éprouvettes possibles sans en affecter la qualité.
    \item	Test: Essai destructif des éprouvettes sur le banc de traction et quantification de la résistance du motif étudié en fonction de la masse, puis de la densité de ce motif.
    \item	Conclusion: Difficulté d'apporter un modèle théorique et paramètres optimaux retenus lors de l'étude.
\end{enumerate}




% Pour faire apparaître la bibliographie avec des chiffres, 
% dans l'ordre d'apparition dans le texte
\bibliographystyle{unsrt-fr}     % Style de la bibliographie (numérotée dans l'ordre d'apparition du texte)
\bibliography{biblio.bib} % Nom du fichier .bib à utiliser
\cite{sports}
\cite{toys}
\cite{moula}
\cite{infill}
\cite{tensiletestd638}
\cite{ASTMD638}
\cite{rdm}
\cite{prusa}
\cite{instron}
\end{document}